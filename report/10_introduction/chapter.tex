This first chapter introduces the interested reader to the subject area of this thesis and its main purposes and demonstrates the demand for research on the covered topics. 

\section{Motivation}
\label{sec:motivation}
Public and academic interest in Autonomous Driving (AD) has grown tremendously over the past decade. As a technology that holds great potential to significantly increase security, efficiency and driver's comfort and to reduce the number of casualties on the road by up to 90 \% \cite{Markwalter2017} it appears to be an inevitable step towards an upcoming revolution in transportation.
Although it is still hard to predict when fully self-driving cars will be publicly available \cite{Frost&SulivanConsulting2018}, technological progress is being achieved at an increasingly rapid pace. Primarily enabled through recent advances in Artificial Intelligence, computation hardware and optical sensor technology, AD systems are continuously becoming more robust and accurate.
\par
\bigskip

However, perception accuracy of today's Advanced Driver Assistance Systems (ADASs) is limited by the range of on-board sensory and a vehicle's line-of-sight (LOS). To be able to safely navigate through complex urban environments, an intelligent vehicle might additionally rely on external observations obtained by surrounding traffic participants, which it constantly exchanges information with through Vehicle-to-Everything (V2X) communication. This concept of combining sensor information across multiple agents to improve perception quality is referred to as Cooperative Perception (CP) and has proven beneficial to address the problem of limited perception and accuracy \cite{Chen2019, Hohm2019}. Its presence is particularly expedient during the market introduction and early adoption phase of self-driving technology, when V2X-enabled cars will face mixed- or predominantly human-controlled traffic.  
\par
\bigskip

Cooperative Perception is believed to hold enormous potential \cite{Gunther2015} and research on related topic is gaining momentum recently \cite{Chen2019, Thandavarayan2019, Calvo2017, BMWGroup2019}. Most current approaches, presented in \cref{ch:related_work}, rely on decentralized, ad-hoc communication and lack a uniform, yet flexible format for representing relevant aspects of a traffic scene. This entails a number of limitations, which are discussed in \cref{ch:problem_analysis}. Moreover, to the best of my knowledge, no holistic CP system has been presented, yet. 
\par
\bigskip

Primary goal of this thesis is to conceptualize, implement and evaluate a comprehensive, end-to-end Cooperative Perception system using novel techniques. Emphasis is laid on the design of a reliable and scalable software architecture and an appropriate schema to model and exchange a shared environment state. Aspects covered in this context include, among others, the suitability and performance of communication via cellular networks, approaches to high-level fusion of time-delayed sensor data and the modeling of uncertain environments. 