As autonomous vehicles are required to operate among highly mixed traffic during their early market-introduction phase, solely relying on local sensory with limited range is potentially not sufficient to exhaustively comprehend and navigate complex urban environments. Addressing this challenge, intelligent vehicles can virtually increase their perception range beyond their line of sight by utilizing V2X communication with surrounding traffic participants to perform cooperative perception. Since existing solutions face a variety of limitations, including lack of comprehensiveness, universality and scalability, this thesis aims to conceptualize, implement and evaluate an end-to-end cooperative perception system using novel techniques. A comprehensive yet extensible modeling approach for dynamic traffic scenes is proposed first, which is based on probabilistic entity-relationship models, accounts for uncertain environments and combines low-level attributes with high-level relational- and semantic knowledge in a generic way. Second, the design of a holistic, distributed software architecture based on edge computing principles is proposed as a foundation for multi-vehicle high-level sensor fusion. In contrast to most existing approaches, the presented solution is designed to rely on Cellular-V2X communication in 5G networks and employs geographically distributed fusion nodes as part of a client-server configuration. A modular proof-of-concept implementation is evaluated in different simulated scenarios to assess the system's performance both qualitatively and quantitatively. Experimental results show that the proposed system scales adequately to meet certain minimum requirements and yields an average improvement in overall perception quality of approximately 27 \%.