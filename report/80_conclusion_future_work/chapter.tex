This last chapter aims to conclude the present work. For this purpose, a summary of the contents and results of previous chapters is given first, followed by an outlook on potential future work to complement this thesis.

\section{Summary}
\label{sec:conclusion:summary}
Overall goal of this work was to design and implement an end-to-end concept for a cooperative perception system. Considered aspects range from environment modeling and state representation over communication- and software architecture throughout to sensor fusion and an in-depth evaluation. 
\par
\bigskip

After this goal was motivated in \cref{ch:introduction} and required background knowledge and fundamentals were provided in \cref{ch:background}, \cref{ch:related_work} presented the current state of the art in all relevant research topics and differentiated the present approach from existing solutions. 
\par
\bigskip

\Cref{ch:problem_analysis} aimed to analyze current approaches and their individual limitations and presented a comprehensive set of goals, functional and non-functional requirements for a novel cooperative perception system to address these limitations. It was observed that current approaches usually lack comprehensiveness, standardization and universality as well as scalability. The latter is usually imposed by the usage of dedicated short-range communication in vehicular ad-hoc networks, which subsequently gave rise to employing fundamentally different communication technologies and patterns.
\par
\bigskip

\Cref{ch:concept_design} was all about elaborating an end-to-end concept that is based on novel techniques and aims to overcome previously described limitations. First, a comprehensive, extensible, yet not complete model for dynamic traffic scenes was proposed. It combines low-level attributes with high-level features and relational knowledge in a generic way and was designed to fulfill previously stated requirements. Geo tiling, occupancy grids and probabilistic entity relationship models were introduced as core building blocks.
Second, major communication technologies and topologies were discussed and a detailed comparison between DSRC-based VANETs and client-server architecture utilizing cellular networks was conducted. Several advantages of the latter regarding latency, throughput and network utilization were outlined.
Third, a holistic system architecture was designed in form of a distributed, messaging-based client-server software solution. Edge computing concepts were incorporated for low latency and load distribution and resilience and scalability are facilitated by the novel approach of geographical partitioning.
Eventually, a fundamental concept on how to perform high-level sensor fusion in the context of cooperative perception was developed. In accordance with the previously presented system design, it involves a centralized fusion node and employs the novel concept of doubly-updated merging. 
\par
\bigskip

\Cref{ch:implementation} discussed details about the concrete implementation of the previously presented concepts. All involved software components were described and categorized into server-side components, including message broker and fusion node and on-board, client-side components, including Talky client and simulator bridge. The entire system was implemented in modular way with clear boundaries and strict interfaces with the goal to enable for easy replacement of individual components, e.g. to use a different simulation environment or messaging backend. Moreover, a multitude of technology choices were made. Carla was chosen as a simulator, MQTT was picked as the central pub-sub messaging protocol for communication among different components and Protobuf was chosen to be employed as a highly efficient binary message format. Eventually, all relevant system parameters and their respective purposes and effects were presented. They were classified into simulation- scene and cooperative perception parameters. 
\par
\bigskip

In \cref{ch:evaluation} a two-fold evaluation was conducted to investigate software performance and scalability as well as the system's general suitability for cooperative perception tasks. It was found that the system is able to scale up to meet the requirements, which were previously designed based on realistic assumptions and estimations of urban traffic volumes. However, further optimization steps are still recommendable in order to apply the system to real-world scenarios. With regard to qualitative performance, the evaluation revealed a potential improvement in overall perception quality of 27 \% through cooperative perception using the present system. 

\section{Outlook}
\label{sec:conclusion:outlook}
As the proposed system constitutes a proof-of-concept implementation rather than aiming to be a production-ready software solution, certain crucial features were considered out of scope and several simplifying assumptions were taken to keep the focus. For instance, all aspects related to security, authentication, data integrity and validation were disregarded. Future research, development and optimization effort might complement the present work at different levels, including the following. 

\begin{itemize}
	\item \textbf{Model:} As mentioned earlier, the current model, presented in \cref{sec:concept_design:environment_modeling_state_representation}, was designed to be easily extensible. However, it can not yet be considered complete, so the specification of an exhaustive model, that covers all potentially relevant aspects, is still required. Moreover, the current model only supports two-dimensional environments for the sake of simplicity. Therefore, it would need to be extended in order to support vertically unambiguous scene descriptions, e.g. as it is the case with a highway bridge over a rural road. 
	\item \textbf{Timing:} \Cref{subsec:concept_design:fusion_summary} and \cref{subsec:evaluation:performance_evaluation:results} already insinuated that timing and synchronization are crucial aspects in a CP system, though they were not thoroughly covered in this work. To maintain data consistency and avoid system failures, the problem of imperfectly accurate clocks must be addressed with more advanced techniques, e.g. using designated hardware devices \cite{Rauch2011} or appropriate algorithms \cite{Julier}. 
	\item \textbf{Fusion:} The proposed fusion algorithm is very fundamental and minimalist and comes with certain limitations. They were briefly discussed in \cref{sec:concept_design:fusion}. More advanced techniques, like track-to-track fusion, the combination with extrapolation and prediction of missing or incomplete observations and the integration with, for instance, Markov chain- or Bayesian network models would be desirable in future work. Similarly, a more elaborate, perhaps adaptive way of temporal decay might be added. Minor optimizations, like the use of Dubin curves for calculating spatial distance as an alternative to plain Eucledian distance, can be thought of in addition. 
	\item \textbf{Communication:} A core point of the concept presented in this thesis is the reliance on a client-server architecture with central fusion nodes. However, a lot of research is going on about device-to-device 5G networks, which imply to revert back to VANET-like communication topologies again. Such might be considered as a serious alternative and are interesting to be put into direct comparison with my present approach. 
	\item \textbf{Scalability:} The evaluation conducted in \cref{sec:evaluation:performance_evaluation} pointed out that the current system's scalability is not entirely sufficient, yet. To some extent this is due to its proof-of-concept character. However, there is still much room for improvements even beyond that. A respective set of possible measures was already presented in \cref{subsec:evaluation:performance_evaluation:discussion_conclusion}. It includes (1) algorithmic optimizations, e.g. transitioning to a tensor representation of observations \cite{Petrich2018} to enable them for being processed with GPU acceleration, (2) adaptive parameter optimization and (3) a more sophisticated way of publishing observations, e.g. using relevance estimation \cite{Breu2013} and caching. 
	\item \textbf{Evaluation:} Not only the system itself, but also its evaluation might be further improved in the context of future work. In addition to using a simulator, results gained in real-world tests with real 5G networks and actual, realistic traffic scenes are desirable. Moreover, an in-depth comparison of the present approach with alternative system – e.g. such based on low-level fusion or using DSRC – would be of great interest.
\end{itemize}
\par
\bigskip

In conclusion, the proposed system is a modern end-to-end approach to cooperative perception, involving an elaborate software architecture and a range of novel concepts and techniques. It comes with a proof-of-concept implementation and a modular integration with a state-of-the-art autonomous driving simulator. Experiments showed its suitability to improve individual automated vehicles' perception quality and hence its potential to facilitate performance, reliability and security of autonomous driving in general. With the help of certain suggested optimizations it might find its way to become a core part of connected autonomous cars in the future. Most likely, cooperative perception solutions like the one developed here will play a crucial role during the early market-introduction phase of highly automated cars, which will find themselves having to operate among heavily mixed traffic. 